\documentclass[12pt,letterpaper]{article}

\usepackage{pdfpages}
\usepackage{hyperref}
\usepackage[title, titletoc, toc]{appendix}
\usepackage{graphicx}
\usepackage{xspace}

\begin{document}

% definitions here

\title{Roadmap for DC2 Production}
\author{ .. (add names/affiliations) ...$^1$, C.W.~Walter$^2$ \\
{\small $^1$SLAC National Accelerator Laboratory, $^2$Duke University}}

\date{\today}

\maketitle

\begin{abstract}
  In this note we describe and document the configurations used for
  the DESC D2 data challenge including the validation metrics for
  certification of suitability for the science studies described in
  the 2017 DESC Science Roadmap document~\cite{SRM:2015}.
\end{abstract}

\noindent
\begin{center}
  \fboxsep=5pt \fbox{\begin{minipage}{5.25in} \it This document is in
      initial draft form.  Add your name and expand the text.
    \end{minipage}} 
  \end{center} 
\vspace{0.1in}

\section{Introduction}

At the June 2017 Stony Brook collaboration meeting the SSim group
surveyed and discussed with the analysis working groups (WG) their
needs for DC2 production.  The WGs were presented with a baseline DC2
configuration and then they presented their plans for DC2 analysis
along with requests for additional features or configuration options
which were currently not present.

This document summarizes the consensus list of the needed parameters
and configuration options and serves as a record of the options used
in the production.  Additionally, in this document we specify the
tests used to validate that an initial test sample of produced images
are appropriate.

The document also presents our production strategy of staging the DC2
production along with which tools will be used for each part and a timeline.
 
\section{Planned uses for the DC2 Simulation}

A summary of typical planned analyses and usages by the various
working groups.


\section{DC2 Cosmological Simulation Configuration and Options}

The DC2 cosmological simulations is the “Outer Rim” (4.225Gpc, $10^9$
mass resolution)” simulation with lensing shear applied.  Parameters
of the simulation are specified in Table~\ref{tab:csim-options} below.

\begin{table}[!htb]
  \centering
  \begin{tabular}{| l| l| }
    \hline 
    Parameter                       & Value   \\
    \hline
    Catalog and Image area  & 300 sq deg \\
     \hline
  \end{tabular}
  \caption{Cosmological Simulation Parameters}
  \label{tab:csim-options}
\end{table}


\section{DC2 Catalog and Image Configuration Options}

The options chosen for the DC2 production are for the most part a
further specification or refinement on the specifications of the XXX
table in SRM. 

Table~\ref{tab:image-options} describes the choice of options for
production as decided after consultation and discussion among the
working groups and those coordination the DC2 production.

\begin{table}[!htb]
  \centering
  \begin{tabular}{| l| l| }
    \hline 
    Parameter                       & Value   \\
    \hline
    Catalog and Image area  & 300 sq deg \\
    Bands                             & ugrizy   \\
    Number of Visits per band  &  (ugrizy) = (56, 80,184,184,160,160) \\
    Cadence                         & OpSim WFD with embedded DDF and twinkles fields\\
    Input                              & Outer rim catalog +MW stars model \\
    Sky Model                      & Basic OpSim Model \\
    Dithering                       & Random translational and rotational  \\
    Sensor Effects                & Basic sensor effects (saturation, tree-rings, brighter-fatter) \\
    Readout type                 & E-image and amplifier readout available \\
    \hline
  \end{tabular}
  \caption{Catalog and image production options}
  \label{tab:image-options}
\end{table}

The command files which implement this configuration can be found in
Appendix~\ref{sec:command-file}.  The CatSim query commands can be
found in~ ??? {\bf (Can we document this?)}.

\section{Production Strategies and Timeline}

This section should cover:

\begin{itemize}
\item Expected uses of PhoSim and imSim
\item Planned staging strategy
\item Anticipated timeline
\end{itemize}

[CWW: This document could also list features still needed to be implemented
in imSim etc before production but I anticipate tracking that
externally and using this document when that code is ready]

\section{Features Implemented and the Current DM Feature Set}

\section{Anticipated Needed Computer Resources}

In the DC1 production approximately 20M cpu-hours were used for the
production of the PhoSim image files, Three 75 TB datasets were
finally produced after production.  With 6 bands and 7 times the area
we anticipate a scaling of a factor of approximately 30 in the amount
of data produced.

\section{Output Format and Processing}

A discussion of issues related to dataframes, Qserve etc.

\section{Validation Samples and Tests}

As part of the staging process we will want the analysis groups to run
tests on the output.  We also would like the analysis groups to
implement some tests now on the DC1 data sets.  Those tests can be
discussed here.

\noindent
{\bf SPECIFY SAMPLES AND VALIDATION TESTS HERE}


\section{Timeline}

An expected timeline could go here instead.

\begin{appendices}

\section{Command Files}
\label{sec:command-file}

The basic command file used to run the production can be
included here.

\begin{verbatim}
Lots O\' Commands
\end{verbatim}

\end{appendices}

\begin{thebibliography}{100}

\bibitem{SRM:2015} 
DESC Science Roadmap,
\url{http://lsst-desc.org/sites/default/files/DESC_SRM_V1.pdf}, 2015

\end{thebibliography}

\end{document}
